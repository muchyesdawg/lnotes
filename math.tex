\documentclass{article} % Document class
\usepackage{amsmath}
\usepackage{tikz}
\usepackage{tikz-among-us}
\title{My bs notes}
\author{Makus}
\date{\today}

\begin{document} % Begin document content
\maketitle
% \grouping{<something>} acts similar to a ToC entry for \chapter*{<something>}


\tableofcontents

\section{Introduction}
    \begin{center}
        just random ass notes, separted by sections.
    \end{center}
\section{math}
    \subsection{bases}
        given  \begin{center}
            $n$ = base of number representation\\
            $x$ = number that is represented in base $n$,
        \end{center}
        Then $\dfrac{x}{n}$ shiftes everything in $x$ towards the left by one and $x\cdot n$ shiftes everything to the right.
\section{bit shit}
\subsection{bitwise operator quirks?}
    In C++, every number is represented in binary, or base 2. Thus if anyone were to divide a number by two, then
    \begin{verbatim}
        x=x>>1;//shifting towards the right
    \end{verbatim}
    works the same, but is faster. it also rounds up. Conversely, if anyone were to multiply by two, then 
    \begin{verbatim}
        x=x<<1;//shifting towards the left
    \end{verbatim}
    does the same.
\subsection{IEEE 754}
    This is how floats in C++ are stored. Floats are stored in 32 bit. the first bit is sign.
    The second to ninth bit is exponent, and the 10th to 32th bit is the mantissa. the exponent has
    an offset of 127, because it needs negative exponents.
    \begin{equation}
        E = x^{n-127}
    \end{equation}
    This can also be represented in the following one liner code (assuming the float has been stored into a long $x$with nothing changed)
    \begin{verbatim}
x = 
    \end{verbatim}
    The mantissa is used to denote the 


\section{Newton's iteration}

\end{document} % End document