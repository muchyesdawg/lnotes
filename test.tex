\documentclass{article} % Document class
\usepackage{amsmath}
\usepackage{tikz}
\usepackage{tikz-among-us}
\title{My bs notes}
\author{Makus}
\date{\today}

\begin{document} % Begin document content
\maketitle
% \grouping{<something>} acts similar to a ToC entry for \chapter*{<something>}


\tableofcontents

\section{Introduction}
    \begin{center}
        just random ass notes, separted by sections.
    \end{center}
\section{math}
    \subsection*{bases}
        given  \begin{center}
            $n$ = base of number representation\\
            $x$ = number that is represented in base $n$
        \end{center}
        then $\dfrac{x}{n}$ shiftes everything in $x$ towards the left by one and $x\cdot n$ shiftes everything to the right.
\section{bit shit}
\subsection{bitwise operator quirks?}
    In C++, every number is represented in binary, or base 2.
    
    \begin{verbatim}
        x<<1
    \end{verbatim}
\subsection{IEEE 754}
    This is how floats in C++ are stored. Floats are stored in 32 bit. the first bit is sign.
    The second to ninth bit is exponent, and the 10th to 32th bit is the mantissa. the exponent has
    an offset of 127, because it needs negative exponents.
    \begin{equation}
        E = x^{n-127}
    \end{equation}
    The mantissa is used to denote the 


\section{Newton's iteration}

\end{document} % End document